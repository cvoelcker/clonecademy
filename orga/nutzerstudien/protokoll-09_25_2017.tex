\documentclass[colorback, accentcolor=tud1c, paper=a4]{tudexercise}
\usepackage[ngerman]{babel}
\usepackage[utf8]{inputenc}
\usepackage{multicol}

%opening
\title{Protokol Nutzerstudie 06.09.2017}
\subtitle{Leonhard Wiedmann}
\subsubtitle{BP Clonecademy}
\author{Leonhard Wiedmann}


\begin{document}

\maketitle
\subsection*{Aufgaben}
	\begin{itemize}
	\item Registrierung
  \item Kurs mit Quiz erfolgreich abschließen
  \item Passwort ändern
	\end{itemize}

\subsection*{Probanden}
Es haben 3 Männer und Frauen an der Nutzerstudie teilgenommen.

\subsection*{Probleme}
  Hier sind die Probleme und Anmerkungen der Probanden aufgelistet. Die Zahl in den Klammern zeigt an, wie viele der Probanden es gemerkt haben oder Probleme damit hatten. Die Zahl in [] gibt an welcher issue id es in Github hat.
  \begin{itemize}
		\item Register mit einer Ungültigen Email gibt keinen Fehler.
    \item Deutlicherer "Submit" Button, so wie "next Question" ist verwirrend
		\item Der übergang vom Kurs zum Quiz ist nicht erkennbar.
		\item Das Ende (Feedback) vom Quiz fehlt.
		\item der Text "Your answer is correct/not correct" im Quiz ist bei langen Fragen nicht sofort erkennbar.
		\item View Profile führt direkt zu Settings, was alle Nutzer erst mal verwirrt hat und sich erst nicht sicher waren ob sie richtig sind um das Passwort zu ändern.
		\item Der Passwort Submit Text sollte lauten "Please enter your old password" oder so ähnlich, da Nutzer dort nochmal ihr neues Passwort eingeben wollten.
		\item durch ändern des Passwort wird das Bild des Nutzers gelöscht. [63]
		\item
   \end{itemize}
\end{document}
