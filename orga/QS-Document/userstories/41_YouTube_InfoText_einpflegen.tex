\begin{tabularx}{\textwidth}{|p{.2\textwidth}|X|}
	\hline
	ID & 41\\
	\hline
	Benutzerrolle & Moderator\\
	\hline
	Name & InfoText mit Youtube Video einpflegen\\
	\hline
	Beschreibung & Als Moderator möchte ich die Möglichkeit haben, einen Informationstext mit einem YouTube Video anstelle einer Frage anzulegen. Im Menu "Kurs anlegen/bearbeiten" kann ich dazu als Frage "InfoText (YouTube)" auswählen und einen Informationstext hinterlegen (siehe US 24). Zusätzlich erhalte ich die Möglichkeit einen Link einzutragen, der auf ein YouTube Video verweißt. Dieser wird in der Datenbank gespeichert.\\
	\hline
	Akzeptanzkriterium & Beim Anlegen des Informationstextes wird in Feld angezeigt, in welches ein YouTube Link geschrieben werden kann. Dieser wird beim Speichern darauf überprüft, ob es tatsächlich ein gültiger Video-Link ist und dann gespeichert. Ist es kein gültiger Link, wird eine entsprechende Fehlermeldung angezeigt. \\
	\hline
	Story Points & 4\\
	\hline
	Entwickler & Claas Völcker\\
	\hline
	Umgesetzt in Iteration & 16\\
	\hline
	Tatsächlicher Aufwand & 3 1/2 h\\
	\hline
	Velocity & \\
	\hline
	Bemerkung & \\
	\hline
\end{tabularx}
\vspace{20pt}
