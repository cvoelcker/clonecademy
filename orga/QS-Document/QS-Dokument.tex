\documentclass[accentcolor=tud0b,12pt,paper=a4]{tudreport}

\usepackage[utf8]{inputenc}
\usepackage{ngerman}
\usepackage{parcolumns}
\usepackage{hyperref}

\setlength{\parindent}{0pt}
\setlength{\parskip}{1em}

\newcommand{\titlerow}[2]{
	\begin{parcolumns}[colwidths={1=.15\linewidth}]{2}
		\colchunk[1]{#1:}
		\colchunk[2]{#2}
	\end{parcolumns}
	\vspace{0.2cm}
}

\title{CloneCademy}
\subtitle{Qualitätssicherungsdokument}
\subsubtitle{%
	\titlerow{Gruppe}{%
		Ilhan Simsiki
		<\href{mailto:ilhan.simsiki@stud.tu-darmstadt.de}{ilhan.simsiki@stud.tu-darmstadt.de}>\\
		Leonhard Wiedmann
		<\href{mailto:leonhard.wiedmann@stud.tu-darmstadt.de}{leonhard.wiedmann@stud.tu-darmstadt.de}>\\
		Tobias Huber
		<\href{mailto:tobias.huber@stud.tu-darmstadt.de}{tobias.huber@stud.tu-darmstadt.de}>\\
		Claas Völcker
		<\href{mailto:c.voelcker@stud.tu-darmstadt.de}{c.voelcker@stud.tu-darmstadt.de}>}
	\titlerow{Teamleiter}{Alexander Nagl
		<\href{mailto:alexander.nagl@t-online.de}{alexander.nagl@t-online.de}>}
	\titlerow{iGEM Team TU Darmstadt}{%
		Thea Lotz <\href{mailto:lotz@bio.tu-darmstadt.de}{lotz@bio.tu-darmstadt.de}>\\
		Fachbereich Biologie}
	\titlerow{Abgabedatum}{xx.xx.xxxx}}
\institution{Bachelor-Praktikum SoSe 2017\\Fachbereich Informatik}

\begin{document}

\maketitle
\tableofcontents

\chapter{Einleitung}

\emph{CloneCademy} ist ein Projekt für das iGEM-Team der TU Darmstadt. Die \emph{international Genetically Engineered Machine competition} (iGEM) ist ein internationaler Wettbewerb für Studierende auf dem Gebiet der Synthetischen Biologie.
Dieser wird seit 2003 von der iGEM-Foundation veranstaltet.

Im Rahmen des Wettbewerbs wird eine Online-Lernplattform für das iGEM-Team der TU Darmstadt erstellt. Das Ziel dieser Plattform ist es, durch interaktive Unterrichtseinheiten Prinzipien der Molekularbiologie sowie der syntetischen Biologie zu erlernen, und sowohl eigene Lernfortschritte als auch die anderer Teams, begutachten zu können. Darüber hinaus soll es auch anderen Interessierten (z.B. andere iGEM-Teams, Lehrende an Universitäten und Schulen, etc.) möglich sein, eigene Inhalte einzupflegen und zur Verfügung zu stellen.

\chapter{Qualitätsziele}
\section{Datensicherheit (Security)}

Im Rahmen des Projekts CloneCademy wird eine Webanwendung entwickelt, auf die über das Internet zugegriffen werden kann. Daher ist die Sicherung gegen unbefugten Zugriff und Änderung der Daten der Webseite in diesem Projekt das wichtigste Qualitätsziel. Da CloneCademy sowohl persönliche Daten der Nutzer*innen als auch Metadaten über die Nutzung der Plattform und die Inhalte der einzelnen Lerneinheiten in einer Datenbank speichert, ist es sehr wichtig, dass Internetnutzer*innen keine Daten verändern oder einsehen können, solange sie dazu nicht die benötigten Rechte besitzen.

Die größte Bedrohung geht von bekannten Webangriffen und Misskonfigurationen im Backend einer Webseite aus. Während dieses Projekts wird darauf geachtet, dass wir die Plattform gegen die wichtigsten Sicherheitslücken in einer Webanwendung sichern.

Diese sind\footnote{Paraphrasiert nach der verbreiteten Liste des Open Web Application Security Project (OWASP) \\
	Link zum Download:  \href{https://github.com/OWASP/Top10/raw/master/2017/OWASP\%20Top\%2010\%20-\%202017\%20RC1-English.pdf}{https://github.com/OWASP/Top10/raw/master/2017/OWASP\%20Top\%2010\%20-\%202017\%20RC1-English.pdf}}:
\begin{enumerate}
\item Möglichkeiten zur Ausführung fremden Codes (Injection \& Cross-Site Scripting)
\item Fehler in Authentifizierung und Session-Management
\item Fehlerhafte Zugriffskontrolle
\item Sicherheitsrelevante Fehlkonfiguration
\item Verlust der Vertraulichkeit sensibler Daten
\item Nutzung von Komponenten mit bekannten Schwachstellen
\item Ungenügend geschützte Programmierschnittstelle
\end{enumerate}

Um die Sicherheit der Anwendung zu gewährleisten, folgen wir einem mehrschrittigen Plan.

Ein Entwickler des Teams wird zum Sicherheitsbeauftragten ernannt. Seine Aufgabe ist es, auf die Einhaltung aller Programmierrichtlinien zu achten und Tests zur Validierung aller Schutzziele durchzuführen. Die genaue Vorgehensweise wird im Folgenden detailliert ausgeführt.

Während der ersten Gespräche mit den Auftraggeber*innen wurden Nutzerrollen definiert (Admin, Moderator*in und Nutzer*in). In jeder User Story wird festgehalten, ob und welche Zugriffsbeschränkungen einzelne Module der Webseite oder Daten besitzen. Während der Implementierung der jeweiligen Story achten die zuständigen Entwickler darauf, dass diese Vorgaben eingehalten werden. Um dies zu gewährleisten, werden die zur Verfügung gestellten Sicherheitswerkzeuge der genutzten Frameworks (Django, Django REST, Angular2) verwendet.

Während der Implementierung achten alle Entwickler darauf, die Best Practices im Web Development einzuhalten. Als Referenz dafür werden die Handreichungen des Open Web Application Security Project (OWASP)\footnote{\href{https://www.owasp.org/images/0/08/OWASP_SCP_Quick_Reference_Guide_v2.pdf}{https://www.owasp.org/images/0/08/OWASP\_SCP\_Quick\_Reference\_Guide\_v2.pdf}} verwendet. In den wöchentlichen internen Code-Reviews wird die Qualität des Codes hinsichtlich dieser Richtlinien überprüft.

Um die Qualität des Codes im Python-basierten Backend automatisiert zu testen, wird das verbreitete Werkzeug \emph{bandit}\footnote{\href{https://wiki.openstack.org/wiki/Security/Projects/Bandit}{https://wiki.openstack.org/wiki/Security/Projects/Bandit}} verwendet. Als statisches Analyse-Werkzeug für das Frontend wird \emph{ng2lint}\footnote{\href{https://www.npmjs.com/package/ng2lint}{https://www.npmjs.com/package/ng2lint}} verwendet. Beide Programme unterstützen die Code-Review, indem die gelieferten Meldungen als Grundlage für eine detaillierte Analyse des Codes genutzt werden. Um eine dauerhaft hohe Qualität des Codes zu gewährleisten, werden die Analyse-Tools jede zweite Iteration verwendet. Das Team achtet darauf, alle gefundenen Schwachstellen soweit wie möglich innerhalb der nächsten zwei Iterationen zu beheben und ansonsten eine realistische Einschätzung über die Konsequenzen der gefundenen Fehler an die Auftraggeber*innen zu melden.

Um die tatsächliche Sicherheit des Endproduktes gegen externe Angriffe zu zeigen, wird ein automatisiertes Penetrations-Werkzeug verwendet. Das OWASP empfiehlt hierfür das \emph{Zed Attack Proxy Project}, welches eine Reihe bekannter Angriffe auf die Plattform ausführt und die gefundenen Schwachstellen meldet. Der Sicherheitsbeauftragte führt diese Tests monatlich durch und meldet die Ergebnisse an die jeweiligen Entwickler des betroffenen Moduls, damit diese innerhalb des nächsten Monats behoben werden können. Ist eine Behebung nicht möglich, werden die Konsequenzen dieser Lücke zusammen mit möglichen weiteren Schritten zum Schutz der Anwendung im Betrieb an die Auftraggeber*innen gemeldet.

\section{Bedienbarkeit (Entwurf)}
CloneCademy muss für jede Benutzerrolle (Admin, Moderator*in, Nutzer*in) bedienbar sein. Um dies zu gewährleisten, müssen alle zugängigen Seiten, intuitiv benutzbar gestaltet werden. Die grundlegende Bedienung der Webseite sollte ohne Einführung und Intuitiv bedienbar sein. Die Lernplatform wird vor allem für Schülern*innen und Studenten*innen im International eingesetzt. Es ist deshalb wichtig eine sehr einfache Bedienung zu schaffen, welche aber auch für eine große Seite komplexe Bedienung ermöglicht.

Um dieses Ziel zu erreichen verwenden wir für den grundlegenden Style der Seite das Material Design \footnote{href{https://material.angular.io/}{https://material.angular.io/}} für Angular2 und um Buttons übersichtlicher zu machen werden diese mit Icons \footnote{\href{https://material.io/icons/}{https://material.io/icons/}} versehen. Da eine Bedienbare Oberfläche eher subjektiv ist werden wir außerdem Nutzerstudien durchführen. In den Studien werden die Benutzer ohne Einführung mit der Platform konfrontiert und sollen darin verschiedene Aufgaben erfüllen. Durch ein Tracking des Bildschirms während der Studie das Verhalten erfasst um es später zu reflektieren und durch einen paar Fragen nach der Studie wird die Persönliche Meinung der Probanden*innen zusätzlich erfasst.

Um zum Schluss ein Bedienbares Produkt zu haben, werden die Nutzerstudien in mehreren Iterationen und mit immer wieder neuen Nutzer durchgeführt um Verbessungsvorschläge aus den Vorherigen Studien direkt wieder zu testen.

\section{Veränderbarkeit (Entwurf)}
Für die Webanwendung CloneCademy ist es wichtig, dass die Anwendung im Nachhinein noch veränderbar ist. Es muss möglich sein, neue Inhalte in die Datenbank einzupflegen und auch den Quellcode der Webseite selbst ohne große Mühe erweitern zu können.

Um dieses Ziel zu erfüllen gehen wir wie folgt vor. Auf folgende Punkte wird im Projekt geachtet:
\begin{description}
\item Quellcode
\item Kommentare
\item Refactoring
\item Wiki
\item Codeanalyse Tools
\end{description}

Es wird ein Beauftragter für das Qualitätsziel gewählt, der für alle Rücksprachen oder Fehler der Ansprechpartner ist und dafür sorgt das die oben genannten Punkte wie unten beschrieben durchgesetzt werden.

\textbf{Quellcode}
Beim implementieren der Userstories wird darauf geachtet das ein einheitliches von Anfang an bestimmtes Muster durchgezogen wird. Klassennamen, Funktionen, Variablen, Klammersetzung etc. Zum Beispiel:
\begin{description}
\item Klassennamen beginnen immer mit einem Großbuchstaben und bei mehreren Wörtern als Klassenname, werden diese zusammen geschrieben und jedes weitere Wort beginnt wieder mit einem Großbuchstaben. Beispiele für Klassennamen: DashboardComponent, ModuleComponent
\item Variablen beginnen immer mit einem Kleinbuchstaben und bestehen im besten Fall aus nur einem Wort. Beispiele: name, title, question, courseID
\item Funktionen beginnen wie Variablen mit einem Kleinbuchstaben, bei mehreren Worten wird wie beim Klassennamen jedes weitere Wort mit einem Großbuchstaben angefangen, das erste Wort aber mit einem Kleinbuchstaben. Beispiele: ngOnInit(), login(form), chartHovered(e:any)
\end{description}
Das Muster wurde am Anfang bevor die Umsetzung angefangen hat gründlich durchgesprochen und wird auch von allen Entwicklern eingehalten. Wenn was neues implementiert wird, schaut sich der Beauftragte den Quellcode nochmal gründlich an und weist den Entwickler daraufhin, sich an das abgesprochene Muster zuhalten. Falls später doch noch etwas von einem Entwickler gefunden wird,  wird das verbessert und im dazu gehörigen Slack Channel protokolliert.

\textbf{Kommentare}
Kommentare haben wie der Quellcode ein einheitliches Muster. Der Code wird durchgehend in englisch kommentiert. Dabei wird beachtet das eine korrekte Sprache benutzt wird. Der beauftragte für das Qualitätsziel ist dafür zuständig bei abgeschlossenen Userstories als Abschluss über den Code zu schauen. Bei jeder Änderung von Code bekommen alle Entwickler automatisch eine Meldung was?, wo? Und wann? Etwas geändert wurde. Der Beauftrage hat dann die Aufgabe die Kommentare abzuchecken. Bei fehlenden Kommentaren wird der Entwickler vom beauftragten kontaktiert und darauf hingewiesen diesem nachzugehen. Ansonsten stellt sich der Beauftrage zwei Fragen:
\begin{description}
\item Erleichtern mir die Kommentare das Verständnis für den Quellcode?
\item Würde jemand der sich mit dem Code gar nicht auskennt damit zurecht finden?
\end{description}
Wenn die Kommentare diese Anforderungen nicht erfüllen, teilt der Beauftragte dem Entwickler das mit, damit er es dann nachbessern kann.




\textbf{Refactoring}
Da bei einem Agilen Projekt es dazu kommen kann, dass der Code sehr schnell unübersichtlich werde kann. Wird vor Abnahme der Userstory ein refactoring vom jeweiligen Entwickler durchgeführt. Dies ist unerlässlich und in diesem Prozess werden unbenutzte Codefragmente entfernt und Quellcode Optimierungen durchgeführt. Bei Änderungen werden die Kommentare darauf abgestimmt.


\textbf{Wiki}
Nebenbei wird eine Wiki regelmäßig aktualisiert, wo alle Schnittstellen zwischen Backend und Frontend definiert sind. Das Wiki kann von allen Gruppenmitgliedern editiert werden. Bei großen Änderungen und neuen Einträgen wird der Beauftragte für das Qualitätsziels kontaktiert, dieser schaut sich die Änderungen an und führt gegeben falls Optimierungen durch. Es wird immer darauf geachtet das alles sauber im Wiki geschrieben wird. Spätestens nach dem refactorn wird alles ins Wiki gepflegt.

\textbf{Codeanalyse Tools}
Für die Codeanalyse wird für das Backend und Frontend jeweils die dazugehörigen statischen Codeanalyse Tools verwendet:

\textbf{ng2lint für Angular2}
Dieser statische Codeanalyse Tool wurde auf die IDE installiert. Ng2lint ist ein konfigurierbares Tool und wurde für die spezifischen Konventionen vom Projekt angepasst. Damit alles läuft und es keine Fehlermeldung gibt müssen im Projekt Regeln aufgestellt werden, an die sich alle halten müssen mit dieser statischen Codeanalyse können wir z.B. automatisch überprüfen:
\begin{description}
\item Das alle Pipes innerhalb des Templates definiert sind
\item Alle benutzerdefinierten Attribute von Elementen in den Vorlagen werden als Eingänge, Ausgänge oder Anweisungen deklariert.
\item Alle verwendeten Identifiers sind in den Tamplates in der entsprechenden Symboltabelle enthalten
\end{descripton}

Vorteile
\begin{description}
\item Warnungen erhalten wennn die Best Practices nicht eingehalten werden.
\item Wenn der Stil, die vom Team abgesprochen wurde nicht eingehalten wird.
\item Falscher Angular 2 spezifischer Code
\end{description}

\textbf{django-lint für Django}
Django Lint ist ein statisches Analyse-Tool, das Projekte und Anwendungen überprüft, die das Django Web Development Framework verwenden.

Es berichtet über gemeinsame Programmierfehler und bad code smells, einschließlich der Überprüfung für nullable CharField-Feldtypen, die Verwendung von spröden oder veralteten Django-Features (wie auto_now_add) sowie das Fehlen von empfohlenen Optionen in settings.py. Es zielt darauf ab, die Entwicklung von hochwertigen wiederverwendbaren Django-Anwendungen zu fördern.



\appendix
	\chapter{Anhang}
		(Am Ende des Projekts nachzureichen)\\

\end{document}
