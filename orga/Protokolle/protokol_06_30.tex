\documentclass[colorback, accentcolor=tud1c, paper=a4]{tudexercise}
\usepackage[ngerman]{babel}
\usepackage[utf8]{inputenc}
\usepackage{multicol}

%opening
\title{Protokoll Auftraggebertreffen 30.06.}
\subtitle{Claas A. Völcker}
\subsubtitle{BP Clonecademy}
\author{Claas A. Völcker}


\begin{document}
	
\maketitle

\begin{multicols}{2}

\section{Bericht der letzte Iteration}

\subsection{User Stories}
\begin{itemize}
	\item Kursfortschritt anzeigen
	\begin{itemize}
		\item abgenommen
	\end{itemize}
	\item Kurs fortsetzen
	\begin{itemize}
		\item abgenommen
		\item Anmerkungen: In den Nutzerstudien wurde bemängelt, dass nicht alle Buttons klar plaziert sind
		\item Der start Course Button könnte die volle Kursübersicht ersetzen
	\end{itemize}
	\item Sprache auswählen
	\begin{itemize}
		\item abgenommen
		\item die deutsche Lokalisierung fehlt noch zu Teilen
	\end{itemize}
\end{itemize}

\section{Design Anmerkungen}
\begin{itemize}
	\item Die Fragenbeantwortung dritteln
	\begin{itemize}
		\item 1. Drittel: Dashboard
		\item 2. Drittel: Fragefenster
		\item 3. Drittel: Feedbackfenster (wenn kein Feedback vorhanden, leer)
		\begin{itemize}
			\item Visuelles Feedback zur Frage (z.B. durch Symbole)
			\item Möglichkeit, einen Feedbacktext für korrekte Antworten einpflegen
			\item Feedbacktext sollte Markdown Support haben
		\end{itemize}
	\end{itemize}
	\item Fragentitel sind erwünscht, wenn keine eingeben werden, sollte ein generischer aus dem Modulnamen generiert werden.
\end{itemize}


\section{Vorschlag für neue User Stories}
\begin{itemize}
	\item Drag \& Drop
	\begin{itemize}
		\item Anschauliche Plasmidkarten werden ins Seafile hochgeladen
		\item Drag\&Drop Elemente sollten nicht zu viel Informationen über die Lösung verraten
		\item Beim Klick auf ein Element sollten die Lösungsmöglichkeiten klar ersichtlich sein
		\item Es könnte mehrere richtige Lösungen geben, es wird überlegt, wie das eingepflegt werden kann
		\item Vorschlag zum Einpflegen:
		\begin{itemize}
			\item Mögliche Antwortfelder auswählbar
			\item Mögliche ''Stückchen'' auswählbar
			\item Gültige Kombinationsmöglichkeiten angebe
		\end{itemize}
	\end{itemize}
	\item Markdown Support
\end{itemize}

\section{Zielvereinbarung für die nächste Woche}
\begin{itemize}
	\item Es wird um eine realistische Einschätzung der Machbarkeit von Drag \& Drop bis nächste Woche durch das BP Team gebeten
	\item BP Team diskutiert über die bessere Platzierung von Knöpfen
	\item Markdown Support wird angeschaut
	\item Überarbeitung des Designs der Frageseite ist erwünscht
	\item User Stories
\end{itemize}

\section{Weitere Planung}
\begin{itemize}
	\item Stabilitätstest wären cool
\end{itemize}

\end{multicols}
\end{document}
