\documentclass[colorback, accentcolor=tud1c, paper=a4]{tudexercise}
\usepackage[ngerman]{babel}
\usepackage[utf8]{inputenc}
\usepackage{multicol}

%opening
\title{Protokoll Auftraggebertreffen 15.09.2017}
\subtitle{Claas Völcker}
\subsubtitle{BP Clonecademy}
\author{Claas Völcker}

\begin{document}

\maketitle

\begin{multicols}{2}

\section*{Bericht der letzte Iteration}
\subsection*{User Stories}
\begin{itemize}
	\item US 51 Profilbild hochladen: abgenommen
	\begin{itemize}
		\item Anmerkung: Ein PopUp/Hinweis, der anzeigt, dass man sich noch einmal authentifizieren muss, wäre hilfreich
		\item neue Userstory: Änderungen im Profil autorisieren (z.B. in diesem PopUp altes Passwort eingeben)
	\end{itemize}
	\item US 49 Profile Quickview: abgenommen
	\item US 45 Eigenes Ranking anzeigen: abgenommen
	\begin{itemize}
		\item Der Schlüssel für die Punkte im Ranking muss angepasst werden.
		\item Nach n richtigen Fragen gibt es Zusatzpunkte.
		\item Diese Zusatzpunkte dürfen nicht doppelt vergeben werden. 
	\end{itemize}
	\item US 39 Dashboard Moderator Übersicht: abgenommen
	\item US 50 Anonyme Statistik: abgenommen
	\item US 54 Visible schalten: abgenommen
	\item US 55 Invisible schalten: abgenommen
	\item US 47 Detaillierte Persönliche Statistik: abgenommen
\end{itemize}


\section*{Weitere Planung - Neue User Stories}
\begin{enumerate}
	\item Profiländerungen authentifizieren: siehe oben - Tobias
	\item Persönliche Statistik erweitern um gesamt Fragen richtig/gesamt und eine Legende - Leo
	\item Quiz Feedback: Dies sollte direkt nach jeder Frage angezeigt werden. - Claas
	\item Quiz Fragen und Antworten: Reihenfolge randomisieren - Leo
	\item Popup bei Fragenantwort anzeigen - Tobias
	\item Ranking Berechnung: siehe oben - Claas
\end{enumerate}

\section*{Weitere Ideen}
\begin{itemize}
	\item Profile Page - Ilhan
	\item Started Courses links in der Liste anzeigen - Ilhan
	\item Kategoriestatistik nur richtig beantwortete Fragen anzeigen - wurde gerade repariert
	\item Colorpicker - todo/offen
\end{itemize}

\section*{Noch im Backlog}
\begin{itemize}
	\item Kurs ändern übersichtlicher machen
	\item Passwort Reset anfordern
	\item Kursstatistik für Moderatoren
	\item Update auf den Server schieben (404 Bug finden)
\end{itemize}



\end{multicols}
\end{document}
